%%%%%%%%%%%%%%%%%%%%%%%%%%%%%%%%%%%%%%%%%%%%%%%%%%%%%%%%%%%%%%%%%%%%%%%%%
%
%   LaTeX File for Doctor (Master) Thesis of Tsinghua University
%   LaTeX + CJK     清华大学博士\KH{硕士}论文模板
%   Based on Wang Tianshu's Template for XJTU
%   Version: 1.00
%   Last Update: 2003-09-12
%
%%%%%%%%%%%%%%%%%%%%%%%%%%%%%%%%%%%%%%%%%%%%%%%%%%%%%%%%%%%%%%%%%%%%%%%%%
%   Copyright 2002-2003  by  Lei Wang (BaconChina)       (bcpub@sina.com)
%%%%%%%%%%%%%%%%%%%%%%%%%%%%%%%%%%%%%%%%%%%%%%%%%%%%%%%%%%%%%%%%%%%%%%%%%


%%%%%%%%%%%%%%%%%%%%%%%%%%%%%%%%%%%%%%%%%%%%%%%%%%%%%%%%%%%%%%%%%%%%%%%%%
%
%   LaTeX File for phd thesis of xi'an Jiao Tong University
%
%%%%%%%%%%%%%%%%%%%%%%%%%%%%%%%%%%%%%%%%%%%%%%%%%%%%%%%%%%%%%%%%%%%%%%%%%
%   Copyright 2002  by  Wang Tianshu    (tswang@asia.com)
%%%%%%%%%%%%%%%%%%%%%%%%%%%%%%%%%%%%%%%%%%%%%%%%%%%%%%%%%%%%%%%%%%%%%%%%%
\renewcommand{\baselinestretch}{1.5}
\fontsize{12pt}{13pt}\selectfont

\chapter{摘~~~~要}
\markboth{中~文~摘~要}{中~文~摘~要}

具有复杂流固边界的流动现象广泛出现在化石能源开采、新能源技术开发、化工过程、环境保护等领域,
这类流动现象通常还耦合多组分/多相组分、化学反应等复杂现象。
%常规计算流体力学方法
格子Boltzmann方法(LBM)由于能够高效处理复杂流固边界、容易耦合多相模型模型而成为研究这类问题的有力工具。
LBM在计算上具有天然并行性,非常适合于在近几年新出现的通用图形处理器(GPGPU)上并行实现。

现代图形处理器(GPU)的计算能力和存储带宽远远超过了目前主流的CPU,其性能/价格比和
性能/能耗比相对于传统的基于CPU的计算机集群或多核计算机具有很大优势。

本文利用CUDA GPU的编程技术,编制了LBM的GPU计算程序,模拟了多孔介质内(复杂边界)
单组分单相和多组分多相流动,重点研究了复杂流固边界与多组分LBM模型在GPU上的高效处理方法,
并对程序做了性能测试和分析。

\vspace{1em}

\noindent {\hei 关键词:} \quad 格子Boltzmann方法,通用图形处理器,GPGPU,多相流,渗流

